\documentclass{beamer}

\usepackage[latin1]{inputenc}
\usepackage[ngerman]{babel}
\usepackage{setspace}
\usepackage{amssymb}
\usepackage{amsmath}
\usepackage{hyperref}
\usepackage{pgf}
\usepackage{graphicx}
\usepackage{caption}
\usepackage{animate}

\beamertemplatenavigationsymbolsempty
\author[F. Loewe]{
  \leftline{\textbf{Projektpartner}: Paul Thurner}
  \leftline{\textbf{Betreuer}: Goeran Kauermann}
  \leftline{\textbf{Referent}: Felix Loewe}
}
\title[Internationaler Waffenhandel]{Internationaler Waffenhandel}
\subtitle{Die Anwendung neuer Verfahren der statistischen Netzwerkanalyse}
\institute[\textbf{LMU} -- Insitut f�r Statistik]{Ludwig-Maximilians-Universit�t M�nchen \\ Institut f�r Statistik}
%\logo{\pgfimage[width=2cm,height=2cm]{lmu_logo}}
%\titlegraphic{\includegraphics[width=1cm,height=1cm]{lmu_logo}}			
\date{\today}
\usetheme{Madrid}
\colorlet{beamer@blendedblue}{green!40!black}
\bibliographystyle{plain}

\begin{document}

%%%%%%%%%%%%%%%%%%%%%%%%%%%% Titel %%%%%%%%%%%%%%%%%%%%%%%%%%%%%%%
\begin{frame}
\maketitle	
\end{frame}

%%%%%%%%%%%%%%%%%%%%%%%%%%%Inhaltsverzeichnis%%%%%%%%%%%%%%%%%%%%%%%%%%%%
\begin{frame}
	\tableofcontents
\end{frame}

%%%%%%%%%%%%%%%%%%%%%%%%Einleitung%%%%%%%%%%%%%%%%%%%%%%%%
\section{Einleitung}
\begin{frame}
	\begin{center}
		\huge \textcolor[rgb]{0,0.33,0}{1 Einleitung}
	\end{center}
\end{frame}

\begin{frame}
	\frametitle{Was ist ein Netzwerk?}
	
	\begin{center}
	Netzwerk besteht aus \emph{Akteuren} und ihren \emph{Verbindungen}\\ \vspace{1cm}
	\end{center}
		
	\textbf{Anwendungsgebiete:}
	\begin{itemize}
		\item \textbf{Biologie}: DNA
		\item \textbf{Soziologie}: Freundesnetzwerk, Kollegenkreis
		\item \textbf{Politik}: internationale Beziehungen
		\item \textbf{Informatik}: Internet, Facebook, LAN
	\end{itemize}
	
\end{frame}

%%%%%%%%%%%%%%%%%%%Einf�hrung in die Graphentheorie%%%%%%%
\section{Einf�hrung in die Graphentheorie}
\begin{frame}
	\begin{center}
		\huge \textcolor[rgb]{0,0.33,0}{2 Einf�hrung in die Graphentheorie}
	\end{center}
\end{frame}

\begin{frame}
	\frametitle{Einf�hrung in die Graphentheorie}
	\textbf{Notation:}
	\begin{itemize}
		\item \textbf{$G = (V,E)$} ... ein \emph{Graph}
		\item \textbf{$V = \left\{1,...,N_V\right\}$} ... Menge der \emph{Knoten}
		\item \textbf{$E = \left\{(i,j)|i,j \in V, i\neq j\right\}$} ... Menge der \emph{Kanten}
		\item $A \in N_V \times N_V$ ... eine \emph{Nachbarschaftsmatrix} \\ \vspace{0.2cm}
	$	a_{ij} =
		\begin{cases}
		  1 \text{ , } ij \in E \\
      0 \text{ , } ij  \notin E \\
   \end{cases}
	$
	\end{itemize}
	\vfill
	\textbf{Begriffe:}
	\begin{itemize}
		\item \emph{gerichteter} vs. \emph{ungerichteter} Graph
		\item (In-/Out-) \emph{Degree}
		\item \emph{Dichte}: $den(G) = \frac{|E_G|}{N_V(N_V-1)/2}$
	\end{itemize}

\end{frame}

%%%%%%%%%%%%%%%%%%%%Datensituation%%%%%%%%%%%%%%%%%%%%%%%%
\section{Datensituation}
\begin{frame}
	\begin{center}
		\huge \textcolor[rgb]{0,0.33,0}{3 Datensituation}
	\end{center}
\end{frame}

\begin{frame}
	\frametitle{Datensituation}
	\begin{center}
		\emph{NISAT}-Datenbank (Norwegian Initiative on Small Arms Transfers) von \emph{PRIO} (Peace Research Institute Oslo)
	\end{center}
	\textbf{Kantenliste mit zus�tzlichen Variablen:}
	\begin{itemize}
		\item Correlates of War Code
		\item monet�rer Wert in US\$
		\item Waffentyp
		\item Datenquelle
		\item Jahr
	\end{itemize}
	\textbf{Dimensionen:}
	\begin{itemize}
		\item 239 L�nder
		\item 20 Jahre
		\item 109522 Waffentransaktionen
	\end{itemize}
\end{frame}

%%%%%%%%%%%%%%%%%Deskriptive Analyse%%%%%%%%%%%%%%%%%%%%%%
\section{Deskriptive Analyse}
\begin{frame}
	\begin{center}
		\huge \textcolor[rgb]{0,0.33,0}{4 Deskriptive Analyse}
	\end{center}
\end{frame}

\subsection{Netzwerkma�zahlen}
\begin{frame}
	\frametitle{Netzwerkma�zahlen}
	
	\begin{figure}
		\centering
					\includegraphics[width=0.8\textwidth]{Grafiken/ts_descriptives.png}
					\caption{Netzwerkma�zahlen des Kleinwaffenhandels \\1992-2011}
					\label{fig:ts_descriptives}		
	\end{figure}
	
\end{frame}
\subsection{Degree-Sequenz}
\begin{frame}
	\frametitle{Degree-Sequenz}
	
	
	\begin{figure}
		\centering
			\includegraphics[width=0.7\textwidth]{Grafiken/ts_degree.png}
		\caption{In-/ Out- Degree der L�nder 1992-2011}
		\label{fig:ts_degree}
	\end{figure}
	
\end{frame}
\subsection{Zentrale Akteure}
\begin{frame}[allowframebreaks]
	\frametitle{Zentrale Akteure}
	
	\begin{table}[ht]

\centering

\begin{minipage}[t]{0.48\textwidth}
\tiny
\centering
\begin{tabular}{rlr}
  \hline
 Platz & Land & Exportvol. [Mrd.]\\ 
  \hline
1 & USA & 9.2\\ 
  2 & Italy & 7.9 \\ 
  3 & Germany & 4.6 \\ 
  4 & Brazil & 3.7 \\ 
  5 & Austria & 2.7 \\ 
  6 & United Kingdom & 2 \\ 
  7 & Belgium & 1.8 \\ 
  8 & Switzerland & 1.5 \\ 
  9 & Russia & 1.4 \\ 
  10 & Czech Republic & 1.4 \\ 
   \hline
	\end{tabular}
	\end{minipage}	
\hfill	
\begin{minipage}[t]{0.48\textwidth}	
\centering
\tiny
\begin{tabular}{rlr}
  \hline
 Platz & Land & Importvol. [Mrd.]\\ 
  \hline
1 & USA & 16\\ 
  2 & Germany & 2.3\\ 
  3 & France & 2.3\\ 
  4 & Canada & 1.9 \\ 
  5 & United Kingdom & 1.8\\  
  6 & Saudi Arabia & 1.7\\ 
  7 & Belgium & 1.2\\ 
  8 & Spain & 1.2\\ 
  9 & Australia & 1.2\\ 
  10 & Turkey & 1\\ 
   \hline
\end{tabular}
\end{minipage}
\caption{Summierte Handelswerte der Top-Exporteure und Top-Importeure des Netzwerkes von 1992 bis 2011}
\label{tab:tops}
\end{table}

\framebreak


\begin{figure}
	\includegraphics[width=0.55\textwidth]{Grafiken/ts_tops.png}
	\label{fig:ts_tops}
\end{figure}

\end{frame}
\subsection{Visualisierungen}
\begin{frame}
	\frametitle{Visualisierungen}
	\begin{figure}[ht]
\centering
\animategraphics[scale=0.35, controls]{1}{Grafiken/Cont_Ani/cont}{1}{20}
\caption{Handelsstr�me zwischen den Kontinenten von 1992-2011}
\label{fig:cont}
\end{figure}
\end{frame}

%%%%%%%%%%%%%%%inferentielle Analyse%%%%%%%%%%%%%%%%%%%%
\section{Inferentielle Analyse}
\begin{frame}
	\begin{center}
		\huge \textcolor[rgb]{0,0.33,0}{5 Inferientielle Analyse}
	\end{center}
\end{frame}

\subsection{ERGM - Exponential Random Graph Model}
\begin{frame}
	\frametitle{ERGM - Exponential Random Graph Model}
\end{frame}
\subsubsection{Simulation von Zufallsgraphen}
\begin{frame}
	\frametitle{Simulation von Zufallsgraphen}
\end{frame}
\subsubsection{Sch�tzung der Modellparameter}
\begin{frame}
	\frametitle{Sch�tzung der Modellparameter}
\end{frame}
\subsection{Anwendung des ERGM}
\begin{frame}
	\frametitle{Anwendung des ERGM}
\end{frame}
\subsection{Vergleich mit Gro�waffenhandel}
\begin{frame}
	\frametitle{Vergleich mit Gro�waffenhandel}
\end{frame}

%%%%%%%%%%%%%%%%Fazit%%%%%%%%%%%%%%%%
\section{Fazit}
\begin{frame}
	\begin{center}
		\huge \textcolor[rgb]{0,0.33,0}{6 Fazit}
	\end{center}
\end{frame}

\begin{frame}
	\frametitle{Fazit}
\end{frame}
\end{document}