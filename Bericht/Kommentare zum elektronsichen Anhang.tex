\documentclass[a4paper,ngerman,oneside,titlepage,bibliography=totoc,11pt]{scrreprt}


\pagestyle{headings}

\usepackage{geometry}
\geometry{left=31mm, top=34mm, right=31mm, bottom=34mm}

\renewcommand{\ttdefault}{lmtt}
\linespread{1.06} %font

\usepackage[font=small,labelfont=bf]{caption} %caption, caption-font
\usepackage{subcaption}
\usepackage{amsmath, amsthm, amssymb, amsbsy}
\usepackage{mathtools}
\usepackage{color}
\usepackage{booktabs}
\usepackage{microtype}
\usepackage{natbib}
\usepackage[ngerman]{babel}
\usepackage[utf8]{inputenc} % f�r Umlaute (ansinew Editor, utf8 bei Texniccenter/Texmaker)
\usepackage[T1]{fontenc} % korrekte Trennung von Umlauten

\usepackage{lmodern}
\usepackage[hyphens]{url}
\usepackage{hyperref}
\usepackage{animate}
\usepackage{rotating}
\usepackage{longtable}
\usepackage{setspace}
\onehalfspacing



\begin{document}

\chapter*{Kommentare zum elektronischen Anhang}
Die Inhalte des elektronischen Anhangs werden hier kurz erkl�rt: Er ist in folgende Ordner unterteilt:
\begin{itemize}
	\item \textbf{Arbeiten zum Gro�waffendatensatz:} Dieser Ordner enth�lt den Bericht und den Vortrag der im Rahmen der Vorg�ngerarbeit erstellt wurde. Au�erdem findet sich hier auch eine Vorabversion des Artikels, der die weiterentwickelte Modellierung des Gro�waffendatensatzes beschreibt.
	\item \textbf{Bericht und Vortrag}: Dieser Ordner enth�lt den diesen Bericht und die dazugeh�rigen Vortragsfolien in elektronischer Form.
	\item \textbf{Importdaten}: Dieser Ordner enth�lt den verwendeten Datensatz des PRIO Institutes sowie die verwendeten Datens�tze der exogenen Kovariablen.
	\item \textbf{R-Code}: Dieser Ordner enth�lt den lauff�higen und kommentierten R-Code mit dessen Hilfe die berichteten Analyse erstellt wurden. Zum Einlesen der Prio-Daten sollte die Datei \texttt{read-in.R} verwendet werden. Die Codes zum Einlesen der exogenen Kovariablen finden sich im Unterordner "`Kovariablen"'. Im Unterordner ERGM finden sich verschiedene ausprobierte Modelle, die durch die Datei \texttt{ergm-vorbereitung.R} lauff�hig werden. Die anderen beiden Unterordner enthalten Code mit dessen Hilfe die im Bericht verwendeten Grafiken erstellt werden k�nnen und einige n�tzliche zus�tzliche Funktionen die im Programmcode selbst ausreichend erkl�rt sind.
	\item \textbf{Sonstiges}: Dieser Ordner enth�lt eine Liste der aus dem Datensatz gel�schten Schleifen und der COW country codes.
\end{itemize} 

\end{document}