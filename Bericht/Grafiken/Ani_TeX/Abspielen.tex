
\documentclass{scrartcl}
\usepackage[left=2cm,right=3cm,top=2cm,bottom=2cm,includeheadfoot]{geometry}
%\usepackage{bibgerm}
\usepackage[utf8]{inputenc}
\usepackage{graphicx}
\usepackage{url}
\usepackage[colorlinks, linkcolor=black, urlcolor=blue,citecolor=black ]{hyperref}
\usepackage{tabularx}

\usepackage{listings}
\usepackage{amssymb,amsmath}
\usepackage{xcolor}
\usepackage{float}
\usepackage{transparent}
\usepackage{setspace}
\usepackage{subfig}
\usepackage{multirow}
\usepackage{here}
\onehalfspacing

\usepackage{colortbl}
\usepackage[round]{natbib}
%\usepackage[german]{babelbib}
\usepackage{animate}
\usepackage{media9}
\usepackage{insdljs}



% Hier JS-Code rein, der einfach so ausgeführt wird:
%\begin{insDLJS}[test]{test}{JavaScript}
%var index = 1000;
%\end{insDLJS}

%%% Wichtig:
\usepackage{animate}


\begin{document}

% ---------------------------------------------------------------
% ---------------------------------------------------------------

%\section{Fragestellung}
\normalsize

\begin{center}
Entwicklung der Handelsströme - Kontinente
\end{center}

\begin{figure}[!htb]
\centering
\animategraphics[scale=0.9, controls]{1}{../Cont_Ani/cont}{1}{20}
\end{figure}
Interpretation:
\begin{itemize}
\item Die Knotengröße spiegelt die Anzahl der Exporte und Importe wieder. Europa hat gleichbleibend am meisten Handelsaktivität. Auf zweiter und dritter Stelle stehen Asien und Amerika.

\item Die Kantendicke ist proportional zum Handelswert gewählt. Der Handelswert zwischen den Kontinenten verändert sich von Jahr zu Jahr. Es gibt kein eindeutiges Länderpaar, das den teuersten Handel betreibt.
\end{itemize}


\pagebreak

\begin{center}
Entwicklung der Handelsströme - Regionen
\end{center}

\begin{figure}[!htb]
\centering
\animategraphics[scale=0.9, controls]{1}{../Reg_Ani/reg}{1}{20}
\end{figure}






\end{document}