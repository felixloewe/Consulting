\documentclass{article}
% Damit die Verwendung der deutschen Sprache nicht ganz so umst\"andlich wird,
% sollte man die folgenden Pakete einbinden: 

\usepackage{geometry}
\usepackage{graphicx}
\usepackage{pgfpages}
\usepackage[ngerman]{babel}
\usepackage[utf8]{inputenc}
\usepackage{textcomp}  	 
\usepackage{amsmath,amsfonts,amssymb} 
\usepackage{booktabs}   
\usepackage{hyperref} 	
\usepackage{multirow}
\usepackage{arydshln}

\geometry{a4paper, top=30mm, left=30mm, right=30mm, bottom=35mm, headsep=10mm, footskip=12mm}
\begin{document}

\pagestyle{empty}

\begin{center}
\large{Ludwig-Maximilians-Universit\"at M\"unchen \\
Institut f\"ur Statistik}\\
\end{center}

\begin{figure}[h!]
\begin{minipage}{1.0\linewidth}\centering
\includegraphics[width=2.5cm]{siegel.jpg}
\end{minipage}
\end{figure}

\begin{center}
\large{Einladung zum Vortrag im Rahmen des Statistischen Consultings}\\
\end{center}
\bigskip

\begin{center}
\Large{\textbf{Internationaler Waffenhandel:\\ Die Anwendung neuer Verfahren der statistischen Netzwerkanalyse}}\\
\end{center}
\vspace{1cm}

\noindent
Der Handel mit Kleinwaffen und leichten Waffen ist in vielen Regionen der Welt schlecht kontrolliert, obwohl Kalaschnikow und Co für deutlich mehr Tote in Kriegsgebieten verantwortlich sind als große Waffen, wie Flugzeuge oder Panzer. Im Rahmen des Projekts sollen die Strukturen des internationalen Handels mit Kleinwaffen und leichten Waffen aufgedeckt und mit bereits vorhandenen Erkenntnissen über den Handel mit Großwaffen verglichen werden.\\
%Invasive Beatmung ist eine lebenserhaltende Ma{\ss}nahme f\"ur Intensivpatienten mit einem schweren Lungenversagen. Es handelt sich hierbei nicht um eine kurative Behandlung, sondern lediglich um eine Ma{\ss}nahme, die dem Patienten Zeit zur Regeneration verschafft. Da die k\"unstliche Beatmung keine heilende Wirkung hat, verbessert sie die \"Uberlebenschance eines Patienten nicht. Als Indikator f\"ur ein schweres Lungenversagen erh\"oht sie stattdessen das Sterberisiko. Zudem kann ein intrinsischer Effekt der Beatmung nicht ausgeschlossen werden. Dieser Gesamteffekt der Beatmung soll in unserem Projekt untersucht werden. \\

\noindent
Die \emph{NISAT Database} des \emph{Peace Research Institute Oslo (PRIO)} enthält Daten über den internationalen Handel mit Kleinwaffen und Leichtwaffen von insgesamt 239 verschiedenen Ländern im Zeitraum von 1992 bis 2011. Diese wurden mit neuen Methoden der statistischen Netzwerkanalyse untersucht.\\

\noindent
Im Rahmen des Projekts wurde zunächst eine deskriptive Analyse durchgeführt, die unter anderem zentrale Akteure des Netzwerkes identifizieren konnte. Ein besonderes Augenmerk galt hier der Veränderung der Netzwerkstrukturen über die Zeit. Anschließend wurde mit Hilfe zusätzlicher Kovariablen versucht ein \emph{Exponential Random Graph Model (ERGM)} auf die Netzwerkdaten der einzelnen Jahre anzupassen. Die Ergebnisse der Modellierung wurden abschließend mit denen eines vorhergehenden Projekts über den Handel mit Großwaffen verglichen.
%Im Rahmen des Consulting-Projekts wurde zun\"achst ein Piecewise Exponential Confounder-Modell aufgestellt. Dieses beinhaltet f\"ur das \"Uberleben prognostisch relevante Einflussfaktoren mit teilweise zeitvariierenden Effekten. In dieses Modell wurden anschlie{\ss}end konstruierte Beatmungsvariablen in verschiedenen Modellierungen hinzugenommen. Somit konnten Aussagen zum Einfluss der Pr\"asenz und der Dauer der Beatmung auf das \"Uberleben der Patienten getroffen werden. 
\vspace{1.5cm}
\hrule 
\begin{table}[h]
 \renewcommand{\arraystretch}{1.5}
\begin{tabular}{ll}
\textbf{Datum:} & Mittwoch, 19.08.2015, 11:45 Uhr \\
\textbf{Ort:} &  Raum 144 (Seminarraum), Institut f\"ur Statistik, LMU M\"unchen\\
\textbf{Projektpartner:} & Prof. Dr. Paul W. Thurner\\
&Lehrstuhl für Empirische Politikforschung und Policy Analysis\\
\textbf{Betreuer:} & Prof. Dr. Göran Kauermann \\
\textbf{Referent:} & Felix Loewe\\
\end{tabular}
\end{table}
\hrule


\end{document}