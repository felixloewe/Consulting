

\documentclass{article}
\usepackage[latin1]{inputenc}
\usepackage[ngerman]{babel}
\usepackage{setspace}
\usepackage{amssymb}
\usepackage{amsmath}
\usepackage{hyperref}
\usepackage{graphicx}
\usepackage{rotating}
\usepackage{textcomp}
\begin{document}

Es stehen folgende zusätzliche Daten zu Verfügung:
\begin{itemize}
  \item Polity: Demokratiescore zwischen 0 und 10. Die Differenz zweier Länder wird als Kantenattribut verwendet.
  \item GDP: Bruttoinlandsprodukt der Länder in internationaler Dollar
  \item Conflict: Interne und externe Konflikte. Knotenattribut mit Score zwischen 0-10
  \item CINC: (Composite Index of National Capability) statistisches Mas für nationale Macht zwischen 0 und 1.
  \item Alliance: binäres Kantenattribut; Besteht ein militärisches Bündnis (1:Ja, 0:Nein)
  \item DirectCont: binäres Kantenattribut; Besteht ein direkte Grenze (1:Ja, 0:Nein)
\end{itemize}

Modell von Schmidt: (Ergebnisse beispielhaft am Jahr 1991)
\begin{itemize}
  \item endogene Statistiken: edges + gwodegree(1, fixed=F) + idegree(1) + dsp(0) + esp(0)
	\item exogene Kantenattribute: edgecov(Alliance) + edgecov(DirectCont)  + edgecov(Polity)
  \item exogene Knotenattribute: nodeicov(GDP) + nodeocov(GDP) + nodeicov(CINC) + nodeocov(CINC) + nodeicov(Conflict))
\end{itemize}

Probleme:
\begin{itemize}
\item endogene statistiken: Modell mit den Statistiken von Schmidt degeneriert!
\item exogene Attribute: 
  \begin{itemize} 
  \item ERGM kann mit fehlenden Werten nicht umgehen. Wir haben deswegen nach Absprache mit Christian alle fehlenden Werte auf 0 gesetzt.
  \item Der Datensatz Conflict enthält teilweise mehrere Konflikte pro Jahr und Land. ERGM kann nur einen Wert verwenden. Nach Absprache mit Christian haben wir das Maximum ("schwerster Konflikt") genommen.
  \end{itemize}
\end{itemize}

Wir habenn nun die exogenen Variablen wie Christian aufgenommen und zusätzlich alle endogenen Statistiken von denen wir wissen, dass sie funktionieren (letztes Treffen).
Wir erhalten folgendes Modell:
\begin{itemize}
\item endogene Statistiken: edges + mutual + idegree(1) + esp(1) + dsp(1)
\item Kantenattribute: edgecov(Alliance) + edgecov(DirectCont)  + edgecov(Polity)
\item Knotenattribute: nodeicov(GDP) + nodeocov(GDP)+ nodeicov(CINC) + nodeocov(CINC) + nodeicov(Conflict))
\end{itemize}





\begin{table}[h]
	\centering
	\caption{summary of model fit}
		\begin{tabular}{l|l|l|l}
		
		\hline
		ergm-term 						    & Estimate      & Std.Error   & p-Value 				\\
		\hline
    edges                     & -2.026e+00    & 7.044e-02   &    < 1e-04 ***  \\
    mutual                    &  4.320e+00    & 9.790e-02   &    < 1e-04 ***  \\
    idegree1                  &  4.906e+00    & 8.307e-01   &    < 1e-04 ***  \\
    esp1                      & -4.767e-01    & 3.730e-01   &    0.20128      \\
    dsp1                      & -1.922e-01    & 3.941e-02   &    < 1e-04 ***  \\
    edgecov.AAlliance[[1]]    & -1.507e-02    & 1.882e-02   &    0.42315      \\
    edgecov.ADirectCont[[1]]  &  3.943e-01    & 1.379e-01   &    0.00425 **   \\
    edgecov.APolity[[1]]      & -3.201e-02    & 4.101e-03   &    < 1e-04 ***  \\
    nodeicov.ext_gdp          & -2.683e-05    & 4.064e-05   &    0.50912      \\
    nodeocov.ext_gdp          & -6.482e-05    & 4.159e-05   &    0.11915      \\
    nodeicov.ext_cinc         & -2.677e+00    & 5.948e+00   &    0.65268      \\
    nodeocov.ext_cinc         & -1.103e+01    & 5.956e+00   &    0.06406 .    \\
    nodeicov.ext_conflict     & -2.008e-03    & 1.976e-02   &    0.91906      \\

		\hline
						
		\end{tabular}
	
	
\end{table}



Interpretation of Time Series:
\begin{itemize}
	\item mutual : Positive value suggests that reciprocated ties are likely. Strength of effect weakens between 1991-1997, then stays quite constant. In our context this means that the weapon trade tends to be symmetric.
	\item gwindegree: Negative popularity spread parameter indicates that most actors have simular levels of popularity (the network is not centralized on in-degree). Strength of effect increases over time. In our context this means that the weapon trade network does not tend to have central importeurs. 
	\item esp(1): A negative effect indicates a low degree of clustering. Strength of effect increases until 1996, then stays quite constant. In our context this means that that the weapon trade network does not tend to form small groups.
	\item dsp(1): This parameter relates to the 2-paths in the networks. A negative estimate indicates that 2-paths tend to be closed (triangles are formed). In our context this means that the weapon trade network tends to form triangles (small groups). This contradicts the findings of the interpretation of esp(1). A joint modeling of both parameters could possibly solve this problem?
	
\end{itemize}

\end{document}